\section{Introduzione}
La specifica della Prova Finale (Progetto di Reti Logiche) 2019 è ispirata al metodo di codifica a bassa dissipazione di potenza denominato \textit{Working Zone}.\newline
Tale metodo è pensato per il Bus Indirizzi e si usa per trasformare il valore di un indirizzo, quando questo viene trasmesso, se appartiene a certi intervalli (detti appunto working-zone). Una working-zone è definita come un intervallo di indirizzi di dimensione fissa ($D_{wz}$) che parte da un indirizzo base. All'interno dello schema di codifica possono esistere multiple working-zone ($N_{wz}$).

\subsection{Obiettivo del progetto}
Dati gli indirizzi base delle working-zone e l'indirizzo da codificare, si vuole implementare un componente hardware descritto in VHDL in grado di leggere l'indirizzo da codificare e gli indirizzi base delle working-zone, verificare l'appartenenza dell'indirizzo da codificare a tali working-zone e produrre l'indirizzo opportunamente codificato in uscita.
In pratica, il modulo da sviluppare si comporta come un \textbf{encoder} di indirizzi.

\subsection{Specifica generale}
Si consideri l'indirizzo da trasmettere $ADDR$. Lo schema di codifica implementato è il seguente:
\begin{itemize}
	\item se $ADDR$ non appartiene a nessuna Working Zone \textbf{[WZ MISS]}, verrà trasmesso un bit addizionale ${WZ}_{BIT}=0$ concatenato ad $ADDR$ (${WZ}_{BIT}$ \& $ADDR$, dove \& è il simbolo di concatenazione);
	
		\begin{figure}[!htb]
			\tikzstyle{int}=[draw, fill=black!10, minimum size=4em]
\tikzstyle{init} = [pin edge={to-,thin,black}]

\centering

\begin{tikzpicture}[node distance=6cm,auto,>=latex']
\node [int] (a) {ENCODER};
\node (b) [left of=a,node distance=5cm]{SOURCE};
\node (end) [right of=a, node distance=8cm]{DEST};
\path[->] (b) edge node {$ADDR$} (a);
\draw[->] (a) edge node {$0$ \& $ADDR$} (end);
\end{tikzpicture}
			\caption{codifica di un indirizzo non appartenente a nessuna WZ.}
			\label{fig:addr_no_wz}
		\end{figure}

	\item se $ADDR$ appartiene ad una Working Zone \textbf{[WZ HIT]}, verrà trasmesso ${WZ}_{BIT}=1$ concatenato a due valori ${WZ}_{NUM}$ e ${WZ}_{OFFSET}$, rappresentanti rispettivamente:
	\begin{itemize}
		\item il numero della working-zone al quale l'indirizzo appartiene, codificato in binario;
		
		\item l'offset rispetto all'indirizzo di base della working-zone, codificato come one-hot.
	\end{itemize}

	
	
	\begin{figure}[!htb]
		\tikzstyle{int}=[draw, fill=black!10, minimum size=4em]
\tikzstyle{init} = [pin edge={to-,thin,black}]

\centering

\begin{tikzpicture}[node distance=6cm,auto,>=latex']
    \node [int] (a) {ENCODER};
    \node (b) [left of=a,node distance=5cm]{SOURCE};
    \node (end) [right of=a, node distance=8cm]{DEST};
    \path[->] (b) edge node {$ADDR$} (a);
    \draw[->] (a) edge node {$1$ \& ${WZ}_{NUM}$ \& ${WZ}_{OFFSET}$} (end);
\end{tikzpicture}	
		\caption{codifica di un indirizzo appartenente ad una WZ.}
		\label{fig:addr_in_wz}
	\end{figure}
\end{itemize}

Si considerino 7 bit per l'indirizzo da codificare (quindi indirizzi validi da 0 a 127).\newline
Il numero di working-zone è 8 ($N_{wz}=8$) mentre la dimensione della working-zone è di 4 indirizzi incluso quello base ($D_{wz}=4$). Questo comporta che l'indirizzo codificato sarà composto da 8 bit: 1 bit per ${WZ}_{BIT}$ + 7 bit per $ADDR$, oppure 1 bit per ${WZ}_{BIT}$, 3 bit per codificare in binario a quale tra le 8 working-zone l'indirizzo appartiene, e 4 bit per codificare one-hot il valore dell'offset di $ADDR$ rispetto all'indirizzo base.\newline

\textbf{Esempio}: codifica di $ADDR$=33, sapendo che la WZ 3 ha come indirizzo di base 31.
\begin{figure}[!htb]
	\centering
	\begin{tikzpicture}[box/.style={fill=black!10, rectangle,draw=black, very thick, minimum size=1cm}]
	
	\foreach \i in {0,...,7} {
		\node at (8-\i-1,0.8){\i};
	}
	
	\foreach \y [count=\x] in {1,0,1,1,0,1,0,0} {
		\node[box] at (\x-1,0){\y};
	}

	\draw[decorate,decoration={brace, amplitude=7pt, raise=0pt, mirror}, thick] (0.6,-.6) -- node[below=8pt]{${WZ}_{NUM}$} (3.4,-.6);
	\draw[decorate,decoration={brace, amplitude=7pt, raise=0pt, mirror}, thick] (3.6,-.6) -- node[below=8pt]{${WZ}_{OFFSET}$} (7.4,-.6);
	\draw[->, thick] (0,-1.2) --  node[below=2pt,yshift=-2mm]{${WZ}_{BIT}$} (0,-.7);

	\node[text width=4cm, anchor=west, right] at (9.5,0) {	
		\begin{tabular}{ |c|c| }
		\hline
		\multicolumn{2}{|c|}{\textbf{Codifica ${WZ}_{OFFSET}$}} \\
		\hline
		offset & one-hot\\
		\hline
		0 & 000\textbf{1}\\
		1 & 00\textbf{1}0\\
		2 & 0\textbf{1}00\\
		3 & \textbf{1}000\\
		\hline
		\end{tabular}
	};
\end{tikzpicture}

	\caption{ADDR appartiene alla WZ 3, con offset 2 (in one-hot).}
	\label{fig:es_addr_in_wz}
\end{figure}



\subsection{Interfaccia del componente}
Il componente descritto possiede la seguente interfaccia:

\begin{lstlisting}[basicstyle=\small, language=VHDL]
entity project_reti_logiche is
port (
	i_clk 		: in std_logic;
	i_start 	: in std_logic;
	i_rst 		: in std_logic;
	i_data 		: in std_logic_vector(7 downto 0);
	o_address 	: out std_logic_vector(15 downto 0);
	o_done 		: out std_logic;
	o_en 		: out std_logic;
	o_we 		: out std_logic;
	o_data 		: out std_logic_vector(7 downto 0)
);
end project_reti_logiche;
\end{lstlisting}


In particolare:
\begin{itemize}
	\item \lstinline[columns=fixed]{i_clk} è il segnale di \lstinline[columns=fixed]{CLOCK} in ingresso generato dal Test Bench;
	
	\item \lstinline[columns=fixed]{i_start} è il segnale di \lstinline[columns=fixed]{START} generato dal Test Bench;
	
	\item \lstinline[columns=fixed]{i_rst} è il segnale di \lstinline[columns=fixed]{RESET} che inizializza la macchina pronta per ricevere il primo segnale di \lstinline[columns=fixed]{START};
	
	\item \lstinline[columns=fixed]{i_data} è il segnale (vettore) che arriva dalla memoria in seguito ad una richiesta di lettura;
	
	\item \lstinline[columns=fixed]{o_address} è il segnale (vettore) di uscita che manda l'indirizzo alla memoria;
	
	\item \lstinline[columns=fixed]{o_done} è il segnale di uscita che comunica la fine dell'elaborazione e il dato di uscita scritto in memoria;
	
	\item \lstinline[columns=fixed]{o_en} è il segnale di \lstinline[columns=fixed]{ENABLE} da dover mandare alla memoria per poter comunicare (sia in lettura che in scrittura);
	
	\item \lstinline[columns=fixed]{o_we} è il segnale di \lstinline[columns=fixed]{WRITE ENABLE} da dover mandare alla memoria (=1) per poter scriverci. Per leggere da memoria esso deve essere 0;
	
	\item \lstinline[columns=fixed]{o_data} è il segnale (vettore) di uscita dal componente verso la memoria.
\end{itemize}

\subsection{Dati e memoria}
Il componente dovrà gestire la comunicazione con una memoria ausiliaria sulla quale saranno presenti i dati in ingresso e andranno salvati quelli in uscita.\newline
I dati, ciascuno di dimensione 8 bit, sono memorizzati sulla memoria con indirizzamento al Byte partendo dalla posizione 0.
Anche l'indirizzo da codificare che è da specifica di 7 bit viene memorizzato su 8 bit (il valore dell'ottavo bit sarà sempre zero).

La memoria è così organizzata:
\begin{itemize}
	\item Le posizioni da 0 a 7 sono usate per memorizzare gli otto indirizzi base delle working-zone;
	
	\item La posizione 8 è usata per memorizzare il valore (indirizzo) da codificare ($ADDR$);
	
	\item La posizione 9 è usata per scrivere il valore codificato in uscita.
\end{itemize}

\begin{figure}[!htb]
	\centering
	\tikzstyle{freecell}=[fill=black!10]
\tikzstyle{occupiedcell}=[fill=black!10]

\renewcommand{\llcell}[3]{
	\addtocounter{cellnb}{-#1}
	\setcounter{ptrnb}{0}
	\draw[#2] (0,\value{cellnb}) +(-2.8,-.5) rectangle +(2.8,-.5+#1);
	\draw (0,\value{cellnb}+#1/2-0.5)  node(currentcell) {#3};
}

\renewcommand{\finishframe}[1]{
	\draw[snake=brace, line width=0.6pt, segment amplitude=8pt]
	(-2.8,\value{cellnb}-0.5) -- (-2.8,\value{startframe}-0.5);
	\draw (-5.4cm,\value{cellnb}*0.5+\value{startframe}*0.5-0.7) node
	{\parbox{3cm}{%
	\begin{flushright}
		#1
	\end{flushright}}};
}

\renewcommand{\separator}[1][freecell,very thick]{
	\draw[#1] (0,\value{cellnb}) +(-2.8,-.5) -- +(2.8,-.5);
}

\renewcommand{\cellcom}[1]{
	\draw (3.1,\value{ptrnb}*0.5+\value{cellnb}) node[anchor=west] {#1};
	\addtocounter{ptrnb}{1}
}

\renewcommand{\stackbottom}[1][freecell]{
	\addtocounter{cellnb}{-1}
	\draw[#1] (0,\value{cellnb})
	+(-2.8,-.5) -- +(-2.8,+.5) -- +(2.8,+.5) -- +(2.8,-.5);
	\draw (0,\value{cellnb}) node{...};
}


\begin{tikzpicture}[scale=0.8]
	\small
	\cell[draw=none]{RAM} \cellcom{Indirizzo}
	\startframe
	\cell{Indirizzo Base WZ 0}        \cellcom{0}
	\cell{Indirizzo Base WZ 1}        \cellcom{1}
	\cell{...} \cellcom{...}
	\cell{Indirizzo Base WZ 7} \cellcom{7}
	\cell{Indirizzo da codificare}     \cellcom{8}
	\finishframe{Input del componente}
	\separator
	\cell{Indirizzo codificato} \cellcom{9}
	\stackbottom{}
	
	\draw[->, thick] (-3.8,-7) --  node[left=2pt,xshift=-4mm]{Output del componente} (-2.8,-7);
	
\end{tikzpicture}

	\caption{indirizzi della RAM rilevanti.}
	\label{fig:ram_1}
\end{figure}

% %% stack-example.tex
  %% Copyright 2010 Matthieu Moy <Matthieu.Moy@imag.fr>
  %
  % This work may be distributed and/or modified under the
  % conditions of the LaTeX Project Public License, either version 1.3
  % of this license or (at your option) any later version.
  % The latest version of this license is in
  %   http://www.latex-project.org/lppl.txt
  % and version 1.3 or later is part of all distributions of LaTeX
  % version 2005/12/01 or later.
  %
  % This work has the LPPL maintenance status `maintained'.
  %
  % The Current Maintainer of this work is M. Matthieu Moy.
  %
  % This work consists of the files drawstack.sty and the example file
  % stack-example.tex.

\documentclass{article}

\usepackage{drawstack}

% Use this instead if you don't want colors.
% \usepackage[nocolor]{drawstack}

\title{{\tt drawstack.sty}: Draw execution stack easily in LaTeX}
\author{Matthieu Moy}

\begin{document}
\maketitle

{\tt drawstack} is a LaTeX package to easily draw execution stack
(typically to illustrate assembly language notions), written on top of
TikZ. This file serves as an example of usage of {\tt drawstack}, and
serves as documentation for this package. Read the source code and
comments to see how to use it.

\section{Minimalistic example}

% The main feature of the package is to define an environment
% drawstack.
\begin{drawstack}
  % Within the environment, draw stack elements with \cell{...}
  \cell{First cell}
  \cell{Second cell}
\end{drawstack}

\section{Grouping cells into stack frames}

\begin{drawstack}
  \startframe
  \cell{First cell}
  \cell{Second cell}
  \finishframe{Some stack frame}
  \cell{Not interesting}
  \startframe
  \cell{Next stack frame}
  \cell{Next stack frame}
  \finishframe{Another stack frame}
\end{drawstack}

\section{Stack and Base pointers}

\begin{drawstack}
  \startframe
  % \cellcom writes something on the right-hand side of a cell.
  \cell{loc2} \cellcom{-8(\%ebp)}
  \cell{loc1} \cellcom{-4(\%ebp)}
  % \esp and \ebp are stack pointer and base pointer in Pentium.
  % These macros are simple shortcuts for \cellptr{...}
  \cell{Sauvegarde \%ebp} \esp \ebp
  \cell{@ retour} \cellcom{4(\%ebp)}
  \finishframe{fonction\\ {\tt f}}
  \startframe
  \cell{} \cellcom{8(\%ebp)}
  \cell{}
  \finishframe{fonction\\ {\tt main}}
\end{drawstack}

\section{Padding}

\begin{drawstack}
  \cell{above padding}
  \padding{3}{nothing here}
  \cell{below padding}
\end{drawstack}

\section{Below/Above stack pointer}

\begin{drawstack}
  \cell{Top}
  \cell{Below top}
  % \bcell is just like \cell, but in a different color.
  \bcell{Above bottom} \cellptr{Stack pointer here}
  \bcell{Bottom}
\end{drawstack}

\section{Highlighting some cell}

\begin{drawstack}
  \cell{Uninteresting cell}
  \cell{Interesting cell} \cellround{Yes, this one!}
\end{drawstack}

\section{Structures without a stack structure}

\begin{tikzpicture}
  \draw (3, -1) node (Otm) {
    \begin{tabular}{c}
      Object\\vtable
    \end{tabular}
  };

  \drawstruct{(0,0)}
  \structcell[freecell]{~} \coordinate (Atm) at (currentcell.east);
  \structcell[freecell]{\texttt{@Object.equals()}}
  \structcell[freecell]{\texttt{@code A.m()}}
  \structcell[freecell]{\texttt{@code A.p()}} \coordinate (A) at (currentcell.west);
  \structname{
    \begin{tabular}{c}
      A's vtable
    \end{tabular}
  }

  \drawstruct{(-4,-3)}
  \structcell[freecell]{} \coordinate (Btm) at (currentcell.east);
  \structcell[freecell]{\texttt{@Object.equals()}}
  \structcell[freecell]{\texttt{@code A.m()}}
  \structcell[freecell]{\texttt{@code B.p()}}
  \structcell[freecell]{\texttt{@code B.q()}}
  \structname{B's vtable}

  \draw[->] (Btm) -- (A);
  \draw[->] (Atm) -- (Otm);
\end{tikzpicture}

\section{Structures and stack together}

\begin{tikzpicture}[scale=.8]

  \stacktop{}
  \separator
  \cell{\texttt{p3}}        \cellcomL{11(GB)} \coordinate (p3) at (currentcell.east);
  \separator
  \cell{\texttt{p2}}        \cellcomL{10(GB)} \coordinate (p2) at (currentcell.east);
  \separator
  \cell{\texttt{p1}}        \cellcomL{ 9(GB)} \coordinate (p1) at (currentcell.east);
  \separator
  \cell{\texttt{@P3D.diag}} \cellcomL{ 8(GB)}
  \cell{\texttt{\footnotesize @Object.equals}} \cellcomL{ 7(GB)}
  \cell{\texttt{3(GB)}}     \cellcomL{ 6(GB)} \coordinate (T1) at (currentcell.east);
  \separator
  \cell{\texttt{@P2D.diag}} \cellcomL{ 5(GB)}
  \cell{\texttt{\footnotesize @Object.equals}} \cellcomL{ 4(GB)}
  \cell{\texttt{1(GB)}}     \cellcomL{ 3(GB)} \coordinate (T2) at (currentcell.east);
  \separator
  \cell{\texttt{\footnotesize @Object.equals}} \cellcomL{ 2(GB)}
  \cell{\texttt{null}}      \cellcomL{ 1(GB)}
  \cell[draw=none]{Stack}


  \drawstruct{(5,1)})
  \structcell{z=2,5}
  \structcell{y=2,5}
  \structcell{x=2,5}
  \structcell{.} \coordinate (O1) at (currentcell.west);
  \coordinate (O1l) at (currentcell.south);

  \drawstruct{(9,-3)}
  \structcell{y=1}
  \structcell{x=1}
  \structcell{.} \coordinate (O2) at (currentcell.west);
  \coordinate (O2l) at (currentcell.south);

  \draw[->] (p3) -- (O1);
  \draw[->] (p2) -- (O1);
  \draw[->] (p1) -- (O2);

  \draw[->] (O1l) .. controls (O1 |- T1) .. (T1);
  \draw[->] (O2l) .. controls (O2 |- T2) .. (T2);

  \draw (10,-10) node{Heap};

\end{tikzpicture}



\section{Using tikzpicture instead of drawstack}

% The environment drawstack is basically a syntactic sugar for
%
% \begin{tikzpicture}[#1]
% \stacktop{}
% ...
% \stackbottom
% \end{tikzpicture}
%
% You can use the above syntax for more flexibility.

\begin{tikzpicture}[scale=0.8]
  \small
  \stacktop{}
  \cell{My cell}
  \stackbottom{}
\end{tikzpicture}

\section{Changing style}

{% tikzstyle will be local to this {...}
\tikzstyle{freecell}=[fill=blue!10,draw=blue!30!black]
\tikzstyle{occupiedcell}=[fill=blue!10!orange!10,draw=blue!30!black]
\tikzstyle{padding}=[fill=yellow!20,draw=blue!30!black]
\tikzstyle{highlight}=[draw=orange!50!black,text=orange!50!black]

\begin{drawstack}
  \cell{Uninteresting cell}
  \cell{Interesting cell} \cellround{Yes, this one!}
  \bcell{bcell}
  \padding{2}{Padding}
\end{drawstack}
}

\section{Example: Computing Factorial}

\begin{drawstack}[scale=0.8]
  \startframe
  \cell{N = 1}
  \cell{...}
  \finishframe{fact(1)}
  \startframe
  \cell{N = 2}
  \cell{...}
  \finishframe{fact(2)}
  \cell{$\vdots$}
  \startframe
  \cell{N = 5}
  \cell{...}
  \finishframe{fact(5)}
\end{drawstack}

\end{document}

%%% Local Variables:
%%% mode: latex
%%% TeX-master: t
%%% End:


