\section{Risultati della sintesi}
Il componente è stato sintetizzato con successo. Di seguito vengono riportati alcuni dei risultati ottenuti durante questa fase.

\subsection{Registri sintetizzati}
In Tabella \ref{table:registers} sono elencati i registri sintetizzati.

\begin{table}[!htb]
	\centering
	\setlength{\belowcaptionskip}{-0.5cm}
	
	\begin{tabular}{ |c|c|l| }
		\hline
		\textbf{\# bit} & \textbf{Quantità} & \textbf{Descrizione} \\
		\hline
		16 & 1 & \specialcell{Indirizzo della RAM. \\ Si utilizzano 16 bit per semplificarne la gestione.} \\ \hline
		 8 & 2 & \specialcell{Indirizzo da codificare e indirizzo codificato in uscita. \\Vengono salvati per assicurare la corretta propagazione dei segnali.} \\ \hline
		 4 & 1 & \specialcell{Offset dell'indirizzo (${WZ}_{OFFSET}$).} \\ \hline
		 3 & 1 & \specialcell{Stato presente della macchina.\\ Sono necessari 3 bit per la codifica dei sei stati.} \\ \hline
		 1 & 5 & \specialcell{${WZ}_{BIT}$, segnale di controllo e tre segnali di uscita del componente.}\\
		\hline
	\end{tabular}

	\caption{}
	\label{table:registers}	
\end{table}

\subsection{Area utilizzata sulla FPGA}
Come da scelte progettuali, si è cercato di ridurre al minimo l'area occupata dal componente sulla FPGA target. Nell'ottica di realizzare componenti più complessi, si è ritenuto che questo encoder non dovesse richiedere risorse che andrebbero impiegate per altri scopi.\newline
Dal report fornito dallo strumento di sviluppo Vivado otteniamo le seguenti percentuali di area occupata rispetto alla FPGA target utilizzata.

\begin{table}[!htb]
	\centering
	\setlength{\belowcaptionskip}{-0.5cm}
	
	\begin{tabular}{ |c|c|c|c| }
		\hline
		\textbf{Tipo} & \textbf{Utilizzati} & \textbf{Disponibili} & \textbf{Percentuale} \\
		\hline
		FF & 32 & 269200 & 0.01\%\\ \hline
		LUT & 39 & 134600 & 0.03\%\\
		\hline
	\end{tabular}
	
	\caption{}
	\label{table:area}	
\end{table}


\subsection{Frequenza di funzionamento}
Con un calcolo approssimativo è possibile ottenere la massima frequenza a cui può operare il componente. Si consideri il periodo di clock $T_{clk}$ di $100 ns$ (da specifica). Dal report si ottiene un \textit{Worst Negative Slack} $WNS$ pari a $94.701 ns$. Il minimo periodo utilizzabile risulta:
\abovedisplayshortskip=5pt
\belowdisplayshortskip=8pt
\begin{equation*}
	T_{min} = T_{clk} - WNS = 100 ns - 94.701 ns = 5.299 ns
\end{equation*}


e quindi si ottiene la massima frequenza di funzionamento:

\abovedisplayshortskip=5pt
\belowdisplayshortskip=0pt
\begin{equation*}
	f_{max} = \frac{1}{T_{min}} = \frac{1}{5.299 ns} \approx 188.72 MHz.
\end{equation*}
